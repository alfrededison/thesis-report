\chapter{Conclusions and Future Work}
Mobile application testing brings a lot of challenges. Regular \acrshort{os} updates, platform or device fragmentation and many other aspects contribute to the low quality of many mobile applications. This thesis focuses and described a testing system that applies Robotics technology for improving mobile application testing process.

\section{Conclusions}
\subsection{Thesis contributions}
During the thesis elaboration, I have learned a lot of image processing techniques, robot control likewise application development and testing.
I developed a program implemented the early mentioned testing system. Also, I modify a mobile screen mirroring application for better access of the program.
\newline
By successful going through from easy to complicated tests, the contribution of this thesis includes several aspects listed as below:
	
    \begin{itemize}
		\item[-] The implementation of the automated mobile testing framework using Robotics technology.
		\item[-] The testing result performed on actual devices and applications with general evaluation.
	\end{itemize}

\subsection{Limitation}
Although the system accomplishes almost test cases successfully, the speed of the Image Processing module is not utterly good. This drawback slows down the testing process and produces some unwanted waiting time.

\section{Future Work}
Although experimental results show that the approach in this thesis is reasonably accurate and is promising, many aspects can have further study to improve the quality and practicality:

	\begin{itemize}
		\item[-] Image processing improvement: for faster and more accurate content recognition, we should try modern computer vision technology and artificial intelligence in order to simulate human perception in mobile phone content. 
		\item[-] Robot enhancement: with more sophisticated gestures like multi-finger touch, the robot needs some more upgrade in design to adapt these daily changed necessities.
		\item[-] Miscellaneous fix: the robot should be reinforced with a stable skeleton for the integrity of initial pointer location. In addition, the stylus also needs to be enhanced so that it can point to the small region of the screen.
	\end{itemize}

From the achievement in this thesis together with expected improvement, we hope that Robotics technology can replace human work, especially in mobile application testing industry.